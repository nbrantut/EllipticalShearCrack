\documentclass[11pt]{article}
\usepackage{amsmath, amssymb}
\usepackage{graphicx}
\usepackage{natbib}
\usepackage{hyperref}
\usepackage[vmargin=2.5cm]{geometry}

\title{Typographical corrections to ``The dynamic field of a growing plane elliptical shear crack'', by Paul G. Richard, \emph{Int. J. Solids Structures}, 1973.}

\author{Nicolas \textsc{Brantut}\thanks{Department of Earth Sciences, University College London, London, UK.}}

\begin{document}

\maketitle

The closed-form solution for the radiation and stresses around a growing elliptical shear crack given by \citet{richards73} contain typographical errors that are not obvious at first sight. After careful examination and rederivation of key formulae, \emph{three} corrections are required to properly implement the solution:
\begin{itemize}
\item p.850, immediately following equation (17), the inequality should read:
  \begin{equation}\label{eq:c1}
    \tau\equiv
    c_\mathrm{d}t/R\geq\tau_\mathrm{ws}\equiv(w^2+c_\mathrm{d}^2/c_\mathrm{s}^2)^{1/2}.
  \end{equation}
  (no factor $i$ should appear in the definition of $\tau_\mathrm{ws}$);
\item p.860, the definition of $q_{\sigma\nu}$ should read:
  \begin{equation}\label{eq:c2}
    q_\mathrm{\sigma\nu} = [w\Delta + i(w^2\Sigma^2N^2 + D)^{1/2}]/D.
  \end{equation}
  ($D$ should appear as the denominator, not factor);
\item p.860, the condition for taking upper or lower sign in expressions for $a_4$, $a_5$, $a_6$, $a_8$ and $a_9$ should read:
  \begin{equation}\label{eq:c3}
    (\sigma^2-\nu^2)\sin2\phi \lessgtr 0.
  \end{equation}
  (the upper/lower inequalities should be flipped.).
\end{itemize}

A Matlab implementation of the solution is given in the code accompanying this note, available at \url{https://github.com/nbrantut/Elliptical_shear_crack.git}. The implementation relies on Matlab's function \verb+quadgk+ to numerically evaluate the integrals. The code was checked against \citeauthor{richards73}' plots.

The present note and accompanying code should help spreading the use of \citeauthor{richards73}' solution in community benchmarks and for any other practical use.

\bibliographystyle{apalike}
\bibliography{references}


\end{document}
